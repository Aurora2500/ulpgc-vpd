\documentclass[spanish]{article}
\usepackage{csquotes}
\usepackage[spanish]{babel}
\selectlanguage{spanish}
\usepackage[utf8]{inputenc}
\usepackage{authblk}
\usepackage{amsmath}
\usepackage{amsfonts}

\usepackage[
	backend=biber,
	style=numeric,
]{biblatex}

\usepackage{enumitem}
\usepackage{extarrows}
\usepackage{mathtools}
\usepackage{systeme}
\usepackage{graphicx}
\usepackage{float}
\usepackage{listings}
\usepackage{listingsutf8}

\usepackage{multirow}
\usepackage{minted}

%\graphicspath{ {./img/} }

%\addbibresource{./sources.bib}

\newcommand{\cimg}[2]{
\begin{figure}[H]
	\center
		\includegraphics[width=#2\linewidth]{#1}
\end{figure}
}

\begin{document}

\begin{titlepage}
	\centering
	{\huge\bfseries Practica 2 - Operaciones de máquinas virtuales \par}
	\vspace{1cm}
	{\scshape\Large Aurora Zuoris \par\tt{aurora.zuoris101@alu.ulpgc.es}\par}
	\vspace{3cm}
	{\scshape\large Virtualización y Processamiento Distribuido\par}
	\vspace{1cm}
	{\scshape\large Grado en Ciencias e Ingeniería de Datos\par}
	\vspace{1cm}
	{\scshape\large Escuela de Ingeniería Informática\par}
	\vspace{1cm}
	{\scshape\large Universidad de Las Palmas de Gran Canaria\par}
	\vspace{1cm}
	{\scshape\large \today{} \par}
\end{titlepage}

\newpage

\tableofcontents

\newpage

\section{Introducción}

En este informe se describe como realizar
varias operaciones con maquinas virtuales.
Primero se muestra como se copia una maquina virtual a mano, ocn el fin 
de entender que archivos son necesarios para que una maquina virtual funcione.
luego se muestra como se copia una maquina virtual con
las herramientas diseñadas para tal fin,
Al final se instala una maquina virtual a partir de virt-install.

\section{Desarollo}

\subsubsection{Copia de una maquina virtual a mano}

Una maquina virtual consiste de principalmente dos archivos,
la configuración de la maquina virtual, y la imagen de su disco duro.
Con lo que para realizar una copia, se debe copiar ambos archivos,
y cambiar la configuración para que no haya conflictos.

\begin{lstlisting}[language=bash,breaklines=true]
$ cd /etc/libvirt/qemu/
$ ls
autostart  MinimalFedoraServer39.xml  networks
$ cp MinimalFedoraServer39.xml HandcloneFedoraServer39.xml -v
'MinimalFedoraServer39.xml' -> 'HandcloneFedoraServer39.xml'

$ cd /var/lib/libvirt/
$ ls
boot  dnsmasq  filesystems  images  network  qemu  swtpm
$ cd images/
$ ls
MinimalFedoraServer39.qcow2
$ cp MinimalFedoraServer39.qcow2 HandcloneFedoraServer39.qcow2 -v
'MinimalFedoraServer39.qcow2' -> 'HandcloneFedoraServer39.qcow2'
\end{lstlisting}

El archivo de configuración XML tiene que ser modificado para que no haya conflictos,
en particular, el nombre de la maquina virtual, el camino
hacia la imagen del disco duro, el UUID, y el MAC address de la tarjeta de red. 

\begin{lstlisting}[language=xml,breaklines=true]
9,10c9,10
<   <name>HandcloneFedoraServer39</name>
<   <uuid>5c2a13ed-6b12-4960-a8d1-fcd209f348b1</uuid>
---
>   <name>MinimalFedoraServer39</name>
>   <uuid>5c2a13ed-6b12-4960-a8d1-fcd209f348b0</uuid>
45c45
<       <source file='/var/lib/libvirt/images/HandcloneFedoraServer39.qcow2'/>
---
>       <source file='/var/lib/libvirt/images/MinimalFedoraServer39.qcow2'/>
136c136
<       <mac address='52:54:00:70:1f:7f'/>
---
>       <mac address='52:54:00:70:1f:6f'/>
\end{lstlisting}

Para el UUID y el MAC address, se pueden usar herramientas que generen nuevos,
pero lo único que importa es que sean diferentes, pues simplemente incrementar
un digito de estos tal que sigua siendo valido es suficiente.

Al final, se debe registrar estos nuevos archivos para que sean
reconocidos por libvirt. Esto se hace con virsh define.

\begin{lstlisting}[language=bash,breaklines=true]
virsh # define /etc/libvirt/qemu/HandcloneFedoraServer39.xml 
Domain 'HandcloneFedoraServer39' defined from /etc/libvirt/qemu/HandcloneFedoraServer39.xml

virsh # list --all
 Id   Nombre                    Estado
-----------------------------------------
 -    HandcloneFedoraServer39   apagado
 -    MinimalFedoraServer39     apagado
\end{lstlisting}


\subsubsection{Copia de una maquina virtual con virt-manager}

Hacer una copia en virt-manager consiste en simplemente
seleccionar la maquina virtual que se quiere copiar,
hacerle click derecho y seleccionar ``Clone".

\subsubsection{Copia de una maquina virtual con virt-clone}

Virt-clone es una herramienta que permite clonar maquinas virtuales.
Para clonar una maquina virtual, se debe especificar el nombre de la maquina virtual
original y el nombre de la maquina virtual clonada.

\begin{lstlisting}[language=bash,breaklines=true]
$ virt-clone -o MinimalFedoraServer39 --name VirtcloneFedoraServer39 --auto-clone
Allocating 'VirtcloneFedoraServer39.qcow2'                  |    0 B  00:00 ... 
Allocating 'VirtcloneFedor 16% [==-              ]    0 B/s | 1.7 GB  --:-- ETA 
El clon 'VirtcloneFedoraServer39' ha sido creado exitosamente.
\end{lstlisting}

\subsubsection{Instalación de una maquina virtual con virt-install}

Para installar una maquina virtual con virt-install, primero se debe montar el directorio donde se encuentran las imagenes de las maquinas virtuales.
Entonces consiste de simplemente usar los flags necesarios para configurar la maquina virtual
a lo deseado.

\begin{lstlisting}[language=bash,breaklines=true]
$ mount -t nfs disnas.dis.ulpgc.es:/imagenes imagenes
Created symlink /run/systemd/system/remote-fs.target.wants/rpc-statd.service -> /usr/lib/systemd/system/rpc-statd.service.

$ virt-install -n "Creacion_virt_install" --memory 2048 --vcpus 1 --disk size=10 -c imagenes/fedora/39/isos/x86_64/Fedora-Server-netinst-x86_64-39-1.5.iso -w network=default

Empezando la instalacion...
Allocating 'Creacion_virt_install.qcow2'                    |  16 MB  00:00 ... 
Creando dominio...                                          |    0 B  00:00     
Running graphical console command: virt-viewer --connect qemu:///system --wait Creacion_virt_install
\end{lstlisting}

\section{Pruebas/Validación}

Para validar que las maquinas virtuales funcionan, se debe
ejecutarlas a la vez y verificar que no haya conflictos
ni en su ejecutamiento ni en su red.

\end{document}
