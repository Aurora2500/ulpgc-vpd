\documentclass{article}
\usepackage{csquotes}
\usepackage[spanish]{babel}
\usepackage{authblk}
\usepackage{amsmath}
\usepackage{amsfonts}

\usepackage[
	backend=biber,
	style=numeric,
]{biblatex}

\usepackage{enumitem}
\usepackage{extarrows}
\usepackage{mathtools}
\usepackage{systeme}
\usepackage{graphicx}
\usepackage{float}

\usepackage{multirow}
\usepackage{minted}

%\graphicspath{ {./img/} }

%\addbibresource{./sources.bib}

\newcommand{\cimg}[2]{
\begin{figure}[H]
	\center
		\includegraphics[width=#2\linewidth]{#1}
\end{figure}
}

\begin{document}

\begin{titlepage}
	\centering
	{\huge\bfseries Practica 1\par}
	\vspace{1cm}
	{\scshape\Large Aurora Zuoris \par\tt{aurora.zuoris101@alu.ulpgc.es}\par}
	\vspace{3cm}
	{\scshape\large Virtualización y Processamiento Distribuido\par}
	\vspace{1cm}
	{\scshape\large Grado en Ciencias e Ingeniería de Datos\par}
	\vspace{1cm}
	{\scshape\large Escuela de Ingeniería Informática\par}
	\vspace{1cm}
	{\scshape\large Universidad de Las Palmas de Gran Canaria\par}
	\vspace{1cm}
	{\scshape\large \today{} \par}
\end{titlepage}

\newpage

\tableofcontents

\newpage

\section{Introducción}

En este informe se describe como se configura una máquina Fedora para
la virtualización con KVM.
Se inicia con una instalación limpia de Fedora,
y se configura para que tenga SELinux, un entorno gráfico
con cual trabajar antes de configurar KVM y
varias otras heramientas de virtualización,
terminando con la creación de una máquina virtual
cargada a travez de la red mediante NFS.

\section{Desarollo}

Lo primero que se hace es actualizar la contraseña de root de la de por defecto,
esto se realiza mediante el comando \texttt{passwd}:

\vspace{0.2cm}
\texttt{\$ passwd}
\vspace{0.2cm}

Una vez cambiado, se sigue a configurar SELinux a modo ``enforcing''
Esto se realiza editando el archivo \texttt{/etc/selinux/config} tal que tenga
la línea \texttt{SELINUX=enforcing}
y con el comando \texttt{grubby}. Al final se debe reiniciar el sistema para que
SELinux se active.

\vspace{0.2cm}

\texttt{
\$ vi /etc/selinux/config
}

\texttt{
\$ grubby --update-kernel ALL --args selinux=1
}

\vspace{0.2cm}

A continuación, se instala un entorno gráfico mediante \texttt{dnf},
en este caso se usa Gnome. Despues de instalarlo se activa mediante \texttt{systemctl}:


\vspace{0.2cm}

\texttt{
\$ dnf groupinstall gnome
}

\texttt{
\$ systemctl set-default graphical.target
}

\vspace{0.2cm}

Luego, se instala todos los paquetes necesarios para virtualización:

\vspace{0.2cm}

\texttt{
\$ dnf install libvirt-daemon-kvm qemu-kvm virt-manager \\ virt-viewer guestfs-tools python3-libguestfs virt-top
}

\vspace{0.2cm}

Para terminar la instalación de estas,
se activa los módulos de kernel de virtualización con \texttt{modprobe}:


\vspace{0.2cm}

\texttt{
\$ modprobe kvm
}

\texttt{
\$ modprobe kvm\_intel
}

\vspace{0.2cm}

La imagen con la que se instala la primera máquina virtual está en un servidor
de la ULPGC, con lo que lo primero que hay que hacer es montar este mediante NFS,
instalando las utilidades necesarias para usar NFS:

\vspace{0.2cm}

\texttt{
\$ dnf install nfs-utils
}

\texttt{
\$ mkdir imagenes
}

\texttt{
\$ mount -t nfs 10.22.146.215:/imagenes imagenes
}

\vspace{0.2cm}

A continuación, se usa el \texttt{virt-manager}
para instalar la imagen de Fedora como una máqunia virtual.
Es imprescindible usar los paquetes predefinidos ante los de la
última version, ya que esto se ha visto a causar problemas tales que la máquina
virtual no arranca correctamente y entra al menú GRUB.
Una vez instalada la máqunia virtual, se pueden actualizar
sus paquetes con \texttt{dnf} para que sean de la última versión.


\section{Pruebas/Validación}

Para comprobar que la máquina es capaz de virtualizar,
se busca dentro de \texttt{/proc/cpuinfo} por el texto \texttt{vmx} o \texttt{svm},
ya que estos indican que el procesador tiene flags activos para la virtualización:

\vspace{0.2cm}

\texttt{
\$ grep -E "vmx|svm" /proc/cpuinfo
}

\vspace{0.2cm}

Además se puede verificar que los módulos del kernel
han sido activados mediante \texttt{lsmod}:

\vspace{0.2cm}

\texttt{
\$ lsmod | grep "kvm"
}

\vspace{0.2cm}

Por último, dentro de la máqunia virtual se pueden usar los
comandos \texttt{free -h}, 
\texttt{df -h}, \texttt{nproc} y \texttt{ping}, 
para comprobar que la memoria RAM, el tamaño del disco virtual,
el número de processos, y la conectividad a internet son adecuados.


\end{document}
